% !TEX encoding = UTF-8 Unicode

\documentclass{ltjsarticle}

\usepackage{graphicx}
\usepackage[deluxe]{luatexja-preset}
\usepackage{luatexja-otf}
\usepackage{url}
%\usepackage{kougai}

\title{少子化によるVRデートアプリの提案について}
\author{1932158 小池 周平  指導教員 須田 宇宙 准教授}
\date{2022年8月3日}

\begin{document}

\maketitle

\section{はじめに}


%背景

2015年9月に150カ国を超える世界のリーダーが参加して開かれた「国連持続可能な開発サミット」でSDGsと呼ばれる「持続可能な開発目標」が掲げられた.このSDGsの中に近年,日本で大きな問題となっている少子化も含まれている.
この大きな問題は,解決するのが遅れれば遅れていくほど悪化し,早急に解決しなくてはならない事象かと思われる.

%問題点
現在少子化には,様々な問題点が存在している.
晩婚化の進展,養育費の高騰,交際率,婚姻率の減少などがあげられる.
交際率,婚姻率が減少していることから未婚者が結婚意欲をそがれているのではないかと思い調査したが図\ref{fig:教科書}にあるように男性・女性ともに85%がいずれは結婚したいと答えており意欲が大きく下がっているわけではないと確認できた\cite{suda2018} .
次に交際率,婚姻率の減少がどこからきているのかについて考える.その理由として交際への不安というものが取り上げられる.
さらに交際への不安がどこからきているのかを考えると出会いの場が少ないという点と時間がないという二つがあげあれる.
出会いの場自体の例を挙げるとすると,学校や職場,サークルや部活,友人づての紹介,マッチングアプリなどが挙げられる.この中で学校,職場,サークル,部活は出会った相手が自分の趣味や嗜好と会うかはわからず,出会いを目的としている人からすると自分と合う人物と巡り合える確率は低いと思われる\cite{suda2020}.
学校や職場などで出会った人と仮に交際まで行けたとしてもその後折り合いが合わなくなり別れることになったとき,別れた後でも顔を合わせる機会があることから付き合うこと自体がリスクにもなりうる.
それらのことから出会いを求めているのであれば相手の嗜好が出会う前から知ることができ,別れたあとのリスクを負うことのないマッチングアプリが最も適しているかと考えられる.
しかし,現在存在しているマッチングアプリのイメージはあまり良いものではなく真剣に交際を考えていて,これから初めてマッチングアプリを利用する者にとっては不安が大きいのではと考えた\cite{suda2021}.
また,出会いの場が減少していることからデート未経験者にとってデートというものに大きな壁を感じ交際の第一歩を踏み出せないということもあるかとも考えられる.
そこで,これらを全て解決し交際について前向きになる方法を探っていく.
初めて会うのが怖いというものは電話やWeb会議,VRなどを駆使すれば解決するかと思われる
デートが初めてでどういったようにすればいいか分からないという方には何かしらのシステムで行く場所の指定をすれば外すリスクは少ないのではないかと思われる.
一度会ってから合うか合わないかの確認,時間がない\cite{suda2019}等も仮想的な空間で短時間で体験できるシステムであれば解決するだろうと考えられる.
これらの解決策から私はVRデートアプリケーションというものを提案する.
\begin{figure}[h]
\begin{center}
 \includegraphics[width=85mm]{fig1.pdf}
\end{center}
 \caption{いずれ結婚したいと思っている18歳から34歳の未婚者の割合}
 \label{fig:教科書}
\end{figure}
%目的

\section{VRデートアプリについて}
アプリケーションの作成自体はUnityで行う.
このアプリケーションはまだデートをしたことがない方,初対面で会うのが怖いと思っている方に向けたアプリケーションである.
アバターは自分の容姿と似ているものを用意し,VR空間を利用して疑似的なデートをするというものである.
仮想的な場所でありながらお互いを認知することができ,場所はシステム側で設定し,40分から1時間程度という短い時間でデートをすることが可能である.

アプリケーションの実装において必要なものは
・上記で挙げたように場所を水族館や港のようなアトラクション向きかつデートに相応しい雰囲気の場所に設定できるようにし,40分から1時間程度と短時間で出来るアプリケーション
・進行は自動で進むものであり遊園地のアトラクション型のようなもの.
・また,相手の表情や共感を得るために表情の実装,手の動きを画面の中と連動させるといったことも必要になってくる.\ref{fig:イラスト}
\begin{figure}[h]
\begin{center}
 \includegraphics[width=85mm]{fig2.pdf}
\end{center}
 \caption{VRアプリの仮定イラスト}
 \label{fig:イラスト}
\end{figure}
%目的

今現在流行しているVRchatとの相違点は,VRchatは自由度が高く出来る事が多いためデート初心者にとっては何処に行けば良いのか,何をしたら良いのかの選択肢が広く計画を立てるのが困難である.そこでこのアプリケーションでは初デートへのハードルを下げることを考えて40分から1時間程度の時間制限を設けアトラクション式の視点移動型のVRを考えている.VRchatと比較した際にデートに特化しており気楽にデートをすることができる.

このアプリケーションの使用前と使用後でそれぞれアンケートを行いユーザーが交際に対して前向きになることができたのか,デートへの壁がなくなったのかを調査する.

現時点で行ったことは類似アプリケーションの確認である.本研究テーマと同じアプリのものがあるのかを確認したが現在のところ同じ目的を持ったアプリケーションは存在していない.




\section{今後の予定}
4年生の内にデートができ,アプリ内で会話が行えるようなシステム開発を行う.
大学院1年生では,アンケート項目を決めるための予備調査やアプリケーションを使用してもらい簡易的な実験を行う.
大学院2年生では,実験を重ね,調査を行い,論文を作成し,学会で発表することも視野に入れている.

\begin{thebibliography}{99}
\bibitem{suda2018} 内閣府: ``少子化対策の現状'', \url{https://www8.cao.go.jp/shoushi/shoushika/whitepaper/measures/w-2016/28webhonpen/html/b1_s1-1-3.html}, 2019/3/26参照
\bibitem{suda2019} 趙 形: ``雇用形態が男性の結婚に与える影響'', 「人口学研究」,第50号,pp75-89(2014)
\bibitem{suda2020} 佐々木 尚之: ``不確実な時代の結婚-JGSSライフコース調査による潜在的稼得力の影響の検証'', 「家族社会学研究」,第24号,pp152-164(2012)
\bibitem{suda2021} 古村 健太郎: ``成人のマッチングアプリ利用の背景-成人のマッチングアプリ利用に関する研究'', 「日本心理学会大会発表論文」,第84号,pc-006(2020)
\end{thebibliography}

\end{document}
